


\documentclass[a4paper]{article}

\usepackage[utf8]{inputenc}
\usepackage[french]{babel}
\usepackage{graphicx}
\graphicspath{{./img/}}
\usepackage{amsmath, amssymb}
\usepackage[left=3cm,right=3cm,top=2cm,bottom=2cm]{geometry}
\usepackage{url}
\usepackage{listings}


\usepackage[french]{babel}
\title{TP n°1 : Entropies discrète et continue, information mutuelle}
\author{Aarab Wassim - Dlimi Mohammed - Ettaki Mohammed Amine}
\date{Décembre 2023}



\begin{document}
\maketitle
\tableofcontents
\newpage
%exo 1 :
\section{Lien entre entropie discrète et continue}
$X$ est une variable aléatoire de densité $f_{X}$ continue.\\
$\forall \Delta > 0$, $X_{\Delta} = \sum_{i\in \mathbb{Z}} x_{i}\mathbb{1}_{[i\Delta,(i+1)\Delta]}(X)$ où $x_i \in [i\Delta,(i+1)\Delta]$.\\

\[
1- \quad
\frac{1}{\Delta} \int_{i\Delta}^{(i+1)\Delta} f_X(x) \, dx = \frac{1}{\Delta} (F_X((i+1)\Delta) - F_X(i\Delta)) = \frac{F_X((i+1)\Delta) - F_X(i\Delta)}{(i+1)\Delta - i\Delta}
\]

D'après le théorème des accroissements finis appliqués sur $F_X$ sur $[i\Delta,(i+1)\Delta]$: \\
\[
\forall i \in \mathbb{Z}, \exists x_i \in ]i\Delta,(i+1)\Delta[, \quad f_X(x_i) = \frac{1}{\Delta} \int_{i\Delta}^{(i+1)\Delta} f_X(x) \,dx
\]

\[
2- \quad
P(X_\Delta = x_i) = P(i\Delta \leq X \leq (i+1)\Delta) = \int_{i\Delta}^{(i+1)\Delta} f_X(x) \, dx = \Delta f_X(x_i) 
\]

\begin{align*}
    3- \quad H(X_\Delta) & = E(-\log(P_X\Delta)) \\
    & = \sum_{i\in \mathbb{Z}} -P(X_\Delta=x_i)\log(P(X_\Delta = x_i)) \\
    & = - \sum_{i\in \mathbb{Z}} \Delta f_X(x_i)\log(\Delta f_X(x_i)) \\
    & = - \Delta\left(\sum_{i\in \mathbb{Z}} f_X(x_i)\log(f_X(x_i)) + \log(\Delta) \sum_{i\in \mathbb{Z}} f_X(x_i)\right) \\
    & = - \Delta\sum_{i\in \mathbb{Z}} f_X(x_i)\log(f_X(x_i)) - \log(\Delta)
\end{align*}


\newpage

%exo 2 Partie 1:
\section{Loi gaussienne}
\subsection{Loi gaussienne univariée.}
on considère maintenant une variable aléatoire réelle $X$ qui suit une loi gaussienne de moyenne $\mu$ et de variance $\sigma^{2}$ (ie. $X \sim \mathcal{N}(\mu, \sigma^{2}))$, on veut maintenant vérifier numériquement le résultat théorique obtenu précédemment pour une variable qui suit une loi gaussienne. Il suffit donc de considérer un nombre très grand de réalisations différentes de $X$ notées $x_i$ pour qu'on puisse définir $X_{\Delta}$ avec un $\Delta$ très petit, car plus le nombre des $x_i$ qu'on choisit est grand plus que la distance entre ses éléments qui est égale à $\Delta$ est petite. On choisit par exemple $n=10000$ réalisations.\\
Pour vérifier maintenant numériquement que la densité de cette loi est bien $f_{X}=\frac{1}{\sigma\sqrt{2\pi}}\exp\left(-\frac{(x-\mu)^{2}}{2\sigma^2}\right)$, on trace l'histogramme de ces $n$ réalisations $x_{i}$ d'une aire normalisée à 1 et la densité $f_{X}$ dans une même figure. On trouve la figure ci-dessous.

\begin{figure}[h]
  \centering
  \includegraphics[width=0.8\textwidth]{Figure_1.png}
  \caption{histogramme-densité}
%  \label{hist}
\end{figure}
ceci prouve que la densité de $X$ est bien $f_X$ puisque les deux graphes sont compatibles,il nous reste maintenant à démontrer le résultat
pour se faire,on calcule numériquement $\mathbb{H}(X_\Delta)+\log(\Delta)\underset{\Delta \rightarrow 0}{\rightarrow}\mathbb{H}(X)$,on calcule tout d'abord le membre gauche de la liit à l'aide d'un code python avec $\delta=1/n$ qui tend vers 0 fourni en annexe,et en testant on trouve la valeur $-4.092594418520442$,ensuite en calcule théoriquement $\mathbb{H}(X)$ à l'aide d'une intégration par partie comme ci dessous:
\begin{align}
  \mathbb{H}(X) &= -\int_{-\infty}^{\infty}f_{X}\log(f_X(x))dx\\
  &=-\int_{-\infty}^{\infty}\frac{1}{\sigma\sqrt{2\pi}}\exp\left(-\frac{(x-\mu)^{2}}{2\sigma^2}\right)\log\left(\frac{1}{\sigma\sqrt{2\pi}}\exp\left(-\frac{(x-\mu)^{2}}{2\sigma^2}\right)\right)\\
  &=-\int_{-\infty}^{\infty}
\end{align}
\newpage

%exo 2 Partie 2:
\subsection{Loi gaussienne multivariée.}
\newpage

%exo3 : 
\section{Analyse de données}

\appendix
\section{Annexe A}
\begin{figure}[h]
  \centering
  \includegraphics[width=0.8\textwidth]{1.png}
  \caption{python histogramme-densité}
%  \label{hist}
\end{figure}

\section{Annexe B}
\begin{figure}[h]
  \centering
  \includegraphics[width=0.8\textwidth]{2.png}
  \caption{val num}
%  \label{hist}
\end{figure}

\end{document}
